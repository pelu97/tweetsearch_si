\newglossaryentry{SI}{
	name={Sistema de Informação},
	description={Uma combinação coesiva de processos que tratam da coleção, trans\-for\-mação, armazenamento e recuperação de dados que contém novidades para usuários, quaisquer que sejam os meios tecnológicos utilizados. Fonte: \citet[p. 38]{prakken_information_2000} (Tradução livre)}}
\newglossaryentry{tecnologia}{
	name={Tecnologia},
	description={Todas as ferramentas, máquinas, utensílios, armas, instrumentos, mora\-dias, vestimentas, dispositivos de comunicação e transporte, combinados com as habilidades pelas quais esses elementos são produzidos e utiliza\-dos. Fonte: \citet[p. 860]{bain_technology_1937}} (Tradução livre)}
\newglossaryentry{SUS}{name={Sistema Único de Saúde},description={É um sistemas público de saúde que abrange desde o simples atendimento ambulatorial até o transplante de órgãos, garantindo acesso integral, universal e gratuito para toda a população do país. Amparado por um conceito ampliado de saúde, o SUS foi criado, em 1988 pela Constituição Federal Brasileira, para ser o sistema de saúde dos mais de 180 milhões de brasileiros. Fonte: \citet{ministerio_da_saude_entenda_2017}}}
\newglossaryentry{VAS}{name={Vigilância Ambiental em Saúde},description={Conjunto de ações e serviços que objetiva o conhecimento, a detecção ou a
prevenção de qualquer mudança em fatores determinantes e condicionantes
do meio ambiente, que possam interferir na saúde humana, no sentido de
recomendar e adotar medidas de prevenção e controle dos fatores de riscos
relacionados às doenças e aos outros agravos à saúde. Fonte: \citet[p. 389]{ministerio_da_saude_o_2009}}}

\newglossaryentry{VS}{name={Vigilância em Saúde},description={A vigilância em saúde abrange as seguintes atividades: a vigilância das doenças
		transmissíveis, a vigilância das doenças e agravos não-transmissíveis
		e dos seus fatores de risco, a vigilância ambiental em saúde e a vigilância
		da situação de saúde. A adoção do conceito de vigilância em saúde procura
		simbolizar uma abordagem nova, mais ampla do que a tradicional prática de
		vigilância epidemiológica. Fonte: \citet[p. 390]{ministerio_da_saude_o_2009}}}
	
\newglossaryentry{Zoonoses}{name={Zoonoses},description={São infecções ou doenças infecciosas transmissíveis, sob condições naturais,
		de homens a animais, e vice-versa. Fonte: \citet[p. 394]{ministerio_da_saude_o_2009}}}

\newglossaryentry{PNCD}{name={Programa Nacional de Controle da Dengue},description={Programa que  tem o objetivo de reduzir
		o número de óbitos e a incidência da doença no Brasil, envolvendo ações
		permanentes e intersetoriais, uma vez que não existem evidências técnicas
		de que, em curto prazo, seja possível a erradicação do mosquito transmissor
		\textit{Aedes aegypti}. Implantado em 2002, desenvolve – em parcerias com estados e municípios – campanhas de comunicação e mobilização social visando
		à prevenção e ao controle do mosquito transmissor (eliminando potenciais
		criadouros do vetor em ambiente doméstico e tratando com larvicidas apenas
		aqueles onde ações alternativas não puderam ser adotadas). Fonte: \citet[p. 114]{ministerio_da_saude_o_2009}}}

\newglossaryentry{Endemia}{name={Endemia},description={Consiste na presença contínua de uma enfermidade ou de um agente infeccioso
		em uma zona geográfica determinada; pode também expressar a prevalência
		usual de uma doença particular em uma zona geográfica. Fonte: \citet[p. 133]{ministerio_da_saude_o_2009}}}

\newglossaryentry{Epidemia}{name={Epidemia},description={Consiste na manifestação, em uma coletividade ou região, de um número de
		casos de alguma enfermidade que excede, claramente, a incidência prevista.
		A quantidade de casos que indica ou não a existência de uma epidemia
		vai variar conforme o agente infeccioso, o tamanho e as características da
		população exposta, sua experiência prévia ou falta de exposição à enfermidade,
		o local e a época do ano em que ocorre. Por decorrência, a epidemia
		guarda relação com a freqüência comum da enfermidade na mesma região,
		na população especificada e na mesma estação do ano. Fonte: \citet[p. 137]{ministerio_da_saude_o_2009}}}

\newglossaryentry{SINAN}{name={Sistema de Informação de Agravos de Notificação},description={Tem por objetivo o registro e o
		processamento dos dados sobre agravos de notificação em todo território nacional, fornecendo informações para análise do perfil da morbidade e contribuindo, dessa forma, para a tomada de decisões nos níveis municipal, estadual e federal. Esse sistema possibilita uma análise global e integrada de todos os agravos definidos para desencadear as medidas de controle. 
		O Sinan é o principal instrumento de coleta de dados das doenças de notificação compulsória e outros agravos. Instituído em 1996, tem por objetivo dotar municípios e estados de uma infra-estrutura tecnológica básica para a transferência de dados dentro de sistema de informação em saúde. Fonte: \citet[p. 461]{ministerio_da_saude_o_2009}}}

\newglossaryentry{DNC}{name={Doenças de notificação compulsória},description={São doenças ou agravos à saúde que devem ser notificados à autoridade sanitária
		por profissionais de saúde ou qualquer cidadão, para fins de adoção de
		medidas de controle pertinentes. Fonte: \citet[p. 124]{ministerio_da_saude_o_2009}}}

\newglossaryentry{SIM}{name={Sistema de Informações sobre Mortalidade},description={Implantado em 1977, entrou em vigor nacionalmente em 1979, permite a obtenção regular de
		dados sobre mortalidade. O documento básico é a Declaração de Óbito (DO). A codificação da
		causa básica do óbito depende do conhecimento de um especialista e, para apoiar esse processo,
		foi desenvolvido o Sistema de Seleção de Causa Básica (SCB). Fonte: \citet[p. 463]{ministerio_da_saude_o_2009}}}

\newglossaryentry{SINASC}{name={Sistema de Informações sobre Nascidos Vivos},description={Implantado oficialmente a partir de 1990, foi desenvolvido à semelhança do Sistema de
		Mortalidade (SIM) com o objetivo de coletar dados sobre nascimentos, em todo território nacional,
		e fornecer dados sobre natalidade para todos os níveis do sistema de saúde. O documento de 
		entrada do sistema é a Declaração de Nascido Vivo (DN), padronizada em todo o País. Fonte: \citet[p. 463]{ministerio_da_saude_o_2009}}}

\newglossaryentry{VE}{name={Vigilância Epidemiológia},description={Ver Vigilância em Saúde. Fonte: \citet[p. 391]{ministerio_da_saude_o_2009}}}


\newglossaryentry{VENT}{name={Vigilância Entomológica},description={Contínua observação e avaliação de informações originadas das características biológicas e ecológicas
dos vetores, nos níveis das interações com hospedeiros humanos e animais
reservatórios, sob a influência de fatores ambientais, que proporcionem
conhecimento para detecção de qualquer mudança no perfil de transmissão das
doenças. Fonte: \citet[p. 82]{gomes_vigilancia_2002}}}

\newglossaryentry{Dengue}{name={Dengue},description={É uma arbovirose transmitida ao homem pela picada do mosquito \textit{Aedes aegypti}. Doença febril aguda, cujo agente etiológico é um vírus do gênero Flavivírus. São conhecidos atualmente quatro sorotipos, antigenicamente distintos: DEN-1, DEN-2, DEN-3 e DEN-4. Clinicamente, as manifestações variam de uma síndrome viral, inespecífica e benigna, até um quadro grave e fatal de doença hemorrágica com choque. Fonte: \citet[p. 100]{tauil_urbanizacao_2001}}}

\newglossaryentry{Arbovirus}{name={Arbovírus},description={Qualquer vírus que é transmitido por artrópodes. Fonte: \citet{ArbovirusWikipedia} (Tradução livre)}}
%\newglossaryentry{}{name={},description={. Fonte: \citet[p. ]{}}
%\newglossaryentry{}{name={},description={. Fonte: \citet[p. ]{}}

